\documentclass[10pt, twocolumn]{article}
\usepackage[landscape, a4paper, margin=2cm]{geometry}
\usepackage[utf8]{inputenc}
\usepackage[T1]{fontenc}
\usepackage{amsmath}
\usepackage{graphicx}


\title{SignSys - Formulaire}

\begin{document}
	\maketitle
	
	\section*{Généralités}
	
		\subsection*{Classement des signaux} 
	
			\begin{itemize}
				\item Continu vs. discret
				\item Déterministe et périodique
				\item Déterministe et transitoire
				\item Aléatoire et stationnaire
				\item Aléatoire et non-stationnaire
				\item Selon l'énergie et la puissance
			\end{itemize}
		
		\subsection*{Puissance et énergie}
		
			Énergie:
			\[W=\sum_{n=0}^{N-1}|x[n]|^{2}\]
			
			Puissance et valeur efficace:
			\[P_{x}=\frac{W}{N}=\frac{1}{N} \sum_{n=0}^{N-1}|x[n]|^{2}\]
			\[P_{x}=P_{d c}+P_{a c}=X_{d c}^{2}+X_{a c}^{2}\]
			\[P_{a c}=P_{x}-X_{d c}^{2}, \quad X_{a c}=\sqrt{P_{a c}}\]
			\[X_{d c}=\frac{1}{N} \sum_{n=0}^{N-1} x[n]\]
			\[X_{r m s}=\sqrt{P_x}=\sqrt{\frac{1}{N} \sum_{k=0}^{N-1}|x[k]|^{2}}\]
			
			Exemples de classement:
			
			\begin{itemize}
				\item Puissance finie, énergie infinie: sinus, saut unité et tous les signaux périodiques
				\item Puissance infinie et énergie finie: impulsion de Dirac
				\item Puissance et énergie infinie: exponentielle
			\end{itemize}
	
	\section*{Signaux et systèmes numériques}
		
		\subsection*{Signaux numériques}
			
			Impulsion unité:
			\[
			\delta[n]=\left\{\begin{array}{lll}
			1 & \text { si } & n=0 \\
			0 & \text { si } & n \neq 0
			\end{array}\right.
			\]
			\[\delta [n] = \epsilon [n] - \epsilon[n-1]\]
			
			Saut unité:
			\[
			\varepsilon[n]=\left\{\begin{array}{lll}
			1 & \text { si } & n \geq 0 \\
			0 & \text { si } & n<0
			\end{array}\right.
			\]
			\[x[n]=\sum_{k=-\infty}^{+\infty} x[k] \cdot \delta[n-k]\]
			
			Exponentielle numérique:
			\[x[n] = R^n \cdot \epsilon [n]\]
			
			Sinusoide:
			\[x[n] = cos(n \Omega_0 + \phi)\]
			\[\Omega_0=\omega_0\cdot T_e = \frac{\omega_o}{f_e} = 2\pi\cdot\frac{f_0}{f_e}=2\pi f_0 T_e \qquad F_0=\frac{f_0}{f_e}=\frac{k}{N}\]
			
			Phaseur de pulsation $\Omega_0$:
			\[x[n] = e^{jn\Omega_0}\]
			
			\subsubsection*{Période numérique}
			
				Période d'échantillonnage: $x[n] = x(n \cdot T_e)$ \\
				Signal périodique: $x[n] = x[n+N]$ \\
				Signal sinusoidal périodique: $N\Omega _0 = k2\pi$
			
		\subsection*{Systèmes numériques}
			
			\subsubsection*{Propriétés des systèmes}
			
				\begin{enumerate}
					\item \emph{Statique:} sans mémoire
					\item \emph{Dynamique:} avec mémoire
					\item \emph{Linéaire:} satisfait au principe de superposition
					\item \emph{Temporellement invariant:} décalage en entrée provoque uniquement un décalage à la sortie
					\item \emph{Causal:} ne dépend que du présent ou du passé
					\item \emph{Stable:} si amplitude finie en entrée, en aucun cas la sortie ne devient infiniment grande
				\end{enumerate}
			
			\subsubsection*{Exemples de quelques systèmes}
				Système identité: $y[n]=x[n]$ \\
				Décalage arrière: $y[n]=x[n-k]$ \\
				Décalage avant: $y[n]=x[n+k]$ \\
				Maximum: $y[n]=\max \{x[n-1], x[n], x[n+1]\}$ \\
				Moyenneur glissant: $y[n]=\frac{1}{5}(x[n]+x[n-1]+x[n-2]+x[n-3]+x[n-4])$ \\
				
				Autres systèmes:
				
				\begin{figure}[h!]
					\includegraphics[width=10cm]{img_sign_sys/exemples_systemes.png}
					\centering
				\end{figure}
				
			
		\subsection*{Réponse impulsionnelle et produit de convolution}
			
			S'applique à:
			\begin{itemize}
				\item Système LIT: linéaire et invariant par translation
				\item Système RIF: à réponse impulsionnelle de durée finie
			\end{itemize}
			
			Produit de convolution:
			\[y[n]=\sum_{k=-\infty}^{+\infty} x[k] h[n-k]=\sum_{k=-\infty}^{+\infty} h[k] x[n-k]\]
			
		\subsection*{Systèmes décrits par des équations récursives}
			
			Accumulateur:
			\[y[n]=y[n-1]+x[n]\]
			
			Filtre passe-bas:
			\[y[n]=\frac{T_{e}}{\tau} x[n]+R y[n-1]\]
			
			Moyenne cumulée:
			\[y[n]=\frac{1}{n+1}(x[n]+n y[n-1])\]
			
		\subsection*{La corrélation}
			
			Signaux à énergie finie:
			\[r_{x y}[k]=\sum_{n=-\infty}^{\infty} x[n] \cdot y[n+k]\]
			
			Auto-corrélation:
			\[r_{x x}[k]=\sum_{n=-\infty}^{\infty} x[n] \cdot x[n+k]\]
			\[r_{x x}[0]=\sum_{n=-\infty}^{\infty} x^{2}[n]=W_{x}\]
			
			Propriétés:
			\[r_{x y}[k]=r_{y x}[-k]\]
			\[r_{x y}[k]=x[-k] * y[k]\]
			\[r_{x x}[k]=r_{x x}[-k]\]
			
			Signaux à puissance moyenne finie:
			\[r_{x y}[k]=\lim _{M \rightarrow \infty} \frac{1}{2 M+1} \sum_{n=-M}^{M} x[n] \cdot y[n+k]\]
			\[r_{x x}[k]=\lim _{M \rightarrow \infty} \frac{1}{2 M+1} \sum_{n=-M}^{M} x[n] \cdot x[n+k]\]
			
			Signaux périodiques:
			\[r_{x y}[k]=\frac{1}{N} \sum_{n=0}^{N-1} x[n] \cdot y[n+k]\]
			\[r_{x x}[k]=\frac{1}{N} \sum_{n=0}^{N-1} x[n] \cdot x[n+k]\]
			
			Normalisation avant corrélation:
			\[x_{n}[n]=\left(x[n]-\mu_{x}\right) / \sigma_{x}\]
			\[y_{n}[n]=\left(y[n]-\mu_{y}\right) / \sigma_{y}\]
			\[\rho_{x y}[k]=\frac{1}{N} \sum_{n=0}^{N-1} x_{n}[n] \cdot y_{n}[n+k]\]
			
	\section*{Réponse des systèmes numériques}
		
		\subsection*{Réponse temporelle des systèmes linéaires}
		\subsection*{Stabilité des systèmes numériques}
		\subsection*{Instants caractéristiques}
			
			\[K_{c}=\pm \frac{1}{\ln (R)}=\frac{1}{|\ln (R)|}\]
			\[K_{p}=\frac{2 \pi}{\Omega}\]
			\[K_{t r} \simeq 5 K_{c}=\frac{5}{|\ln (R)|}\]
			\[N_{osc}=\frac{K_{t r}}{K_{p}}=\frac{5 \Omega}{2 \pi|\ln (R)|} \simeq \frac{\Omega}{|\ln (R)|}\]
	
	\section*{Analyse temporelle des systèmes discrets}
		
		\subsection*{Transformée en Z}
			\[X(z)=Z\{x[n]\}=\sum_{n=0}^{+\infty} x[n] z^{-n}\]
			\[z^{-1} \equiv e^{-s T_{e}}\]
			
			% fig transformées en Z
			
		\subsection*{Propriétés de la transformée en Z}
			
			\begin{enumerate}
				\item Linéarité: $Z\{a x[n]+b y[n]\}=a X(z)+b Y(z)$
				\item Décalage temporel: $Z\{x[n+d]\}=z^{+d} X(z)$
				\item Amortissement: $Z\left\{\alpha^{n} x[n]\right\}=X\left(\frac{z}{\alpha}\right)$
				\item Valeur initiale:$x[0]=\left.X(z)\right|_{z \rightarrow \infty}$
				\item Valeur finale (si système stable): $x[\infty]=\left.(z-1) X(z)\right|_{z=1}$
			\end{enumerate}
			
		\subsection*{Équation aux différences et fonction de transfert}
			
			\[y[n]+\sum_{k=1}^{N} a_{k} y[n-k]=\sum_{k=0}^{M} b_{k} x[n-k]\]
			
			Pour un système d'ordre 2:
			\[y[n]+a_{1} y[n-1]+a_{2} y[n-2]=b_{0} x[n]+b_{1} x[n-1]+b_{2} x[n-2]\]
			
			Forme de réalisation:
			\[H(z) \equiv \frac{Y(z)}{X(z)}=\frac{b_{0}+b_{1} z^{-1}+b_{2} z^{-2}}{1+a_{1} z^{-1}+a_{2} z^{-2}}\]
			
			Forme analytique:
			\[H(z)=\frac{b_{0} z^{2}+b_{1} z+b_{2}}{z^{2}+a_{1} z+a_{2}}\]
			
		\subsection*{Lieu de pôles}
			
			% fig lieu des pôles
			
	\section*{Analyse fréquentielle des signaux continus et discrets}
	
		\subsection*{Analyse des signaux périodiques}
		\subsection*{Transformation de Fourrier des signaux non périodiques}
		\subsection*{Éléments d'analyse spectrale numérique}
	
	\section*{Analyse fréquentielle des systèmes, et synthèse des filtres}
		
		\subsection*{Réponse fréquentielle des systèmes LTI}
		\subsection*{Pôles, zéros et réponse fréquentielle}
		\subsection*{Calcul et traçage de réponse fréquentielle}
		\subsection*{Analyse et réalisation d'un filtre}
		\subsection*{Classification des systèmes numériques}
		
	\section*{Échantillonnage et reconstruction des signaux analogiques}
	
		\subsection*{Introduction}
		\subsection*{Analyse temporelle}
		\subsection*{Analyse fréquentielle}
		\subsection*{Recouvrement spectral}
		\subsection*{Théorème de l'échantillonnage}
		\subsection*{Quantification d'un signal échantillonné}
		\subsection*{Choix d'un filtre et de la fréquence d'échantillonnage}
		\subsection*{Reconstruction du signal}
		\subsection*{Analyse qualitative d'une chaîne A/N - N/A}
		
\end{document}