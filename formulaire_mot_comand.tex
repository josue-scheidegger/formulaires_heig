\documentclass[10pt]{article} %twocolumn
\usepackage[landscape, a4paper, margin=2cm]{geometry}
\usepackage[utf8]{inputenc}
\usepackage[T1]{fontenc}
\usepackage{textcomp}
\usepackage{graphicx}
\usepackage{float}
\usepackage{amsmath}
\usepackage{multicol}

\title{Formulaire MotComand}

\begin{document}
	%\maketitle
	\begin{multicols}{3}
	\begin{flushleft}
		\section*{Mouvement dans les machines}
		
			\subsection*{Actionneurs et moteurs}
	
				\begin{itemize}
					\item \emph{Actionneurs et moteurs pneumatiques:} économiques, faible coûts d'entretien, conviennent aux milieux hostiles, vitesses élevées; temps de réaction < 20ms, bruit, positions limitées (tout-ou-rien).
					\item \emph{Actionneurs et moteurs hydrauliques:} performants, haute densité d'énergie, réglage en vitesse ou en position; coûteux, entretien plus compliqué (huile), temps de réponse d'environ 2ms.
					\item \emph{Moteurs électriques:} économiques, beaucoup de fournisseurs, faciles à mettre en oeuvre, temps de réponse de 0.1ms; nécessitent en général des réducteurs.
				\end{itemize}
				
				Comparaison:
				
				\begin{figure}[H]
					\includegraphics[width=8cm]{img/comparaison_actionneurs.png}
					\centering
				\end{figure}
			
			\subsection*{Constitution des entraînements}
				
				\begin{figure}[H]
					\includegraphics[width=8cm]{img/constitution_entrainements.png}
					\centering
				\end{figure}
				
			\subsection*{Charges}
				
				Lois de Newton:
				\[a = \frac{\sum F}{m}\]
				
				\begin{itemize}
					\item a : accélération $[m/s^2]$
					\item F : forces $[N]$
					\item m : masse $[kg]$
				\end{itemize}
		
				\[\alpha = \frac{\sum T}{J} \]
		
				\begin{itemize} 
					\item $\alpha$ : accélération angulaire $[rad/s^2]$
					\item T : couples $[N \cdot m]$
					\item J : inertie $[kg \cdot m^2]$
				\end{itemize}
		
				Inertie d'un cylindre:
				
				\[ J = \frac{m \cdot R^2}{2} = \frac{\rho \cdot L \cdot \pi \cdot R^4}{2} \]
				
				Quadrants de fonctionnement:
				
				\begin{table}[H]
				\begin{tabular}{|c|c|c|l|}
				\hline
					& $\omega$ & $T_{em}$ & Mode   \\ \hline
				1 & +        & +        & Moteur \\ \hline
				2 & -        & +        & Frein  \\ \hline
				3 & -        & -        & Moteur \\ \hline
				4 & +        & -        & Frein  \\ \hline
				\end{tabular}
				\centering
				\end{table}
				
				Types de charge:
				
				\begin{itemize}
					\item Charge à couple constant
					\item Charge à couple croissant avec la vitesse
					\item Charge à puissance constante:
					\item [] $P(t)=T(t) \cdot \omega(t)=[F \cdot r(t)] \cdot\left[\frac{V}{r(t)}\right]=F \cdot V=$ cste (avec F et V, respectivement force et vitesse tangentielles)
				\end{itemize}
				
				Régimes de fonctionnement:
				
				\[T_{moteur}(t) - T_{resistant}(t) = T_{accel.}(t) \]
				\[T_{resistant}(t) = T_{frott.}(t) + T_{utile}(t) \]
				
				\begin{itemize}
					\item \emph{Régime permanent:} la charge tourne à vitesse (quasi) constante
					\item \emph{Régime impulsionnel ou intermittent:} la charge est constamment accélérée et freinée
				\end{itemize}
	
			\subsection*{Modes de fonctionnement}
	
				\begin{itemize}
					\item \emph{Mode tout-ou-rien:} le plus simple et le plus bon marché; pas d'adaptation à la charge entraînée.
					\item \emph{Mode contrôlé en boucle ouverte:} contrôle approximatif de la vitesse et de l'effort fournis.
					\item \emph{Mode contrôlé en «boucle fermée»:} grande précision; plus complexe et coûteux.
					\item \emph{Mode servomoteur - réglé en position:} permet de contrôler tous les mouvements d'une machine; complexe et coûteux.
					\item \emph{Mode pas-à-pas:} simple et bon marché; limité en puissance (~200W) et vitesse (~1000tr/min).
				\end{itemize}
	
		
	
			\subsection*{Modèle thermique des moteurs}
	
				\[\Delta T = R_{th} \cdot P_{moy} \]
				\[P_c = C_{th} \cdot \frac{dT}{dt} \]
				\[T(t) = (T_{max} - T_0) \cdot (1 - e^{-t/\tau_{th}}) + T_0 \]

				Comparaison thermique-électricité:
	
				\begin{itemize}
					\item Courant électrique [A] = Puissance thermique [W]
					\item Tension électrique [V] = Température [°C]
					\item Capacité électrique = Capacité thermique
				\end{itemize}
	
	\section*{Réducteurs}
		
		Types de réducteurs:
		
		\begin{itemize}
			\item réducteurs rotatifs-rotatifs (le moteur et la charge sont rotatifs)
			\item réducteurs rotatifs-linéaire (le moteur est rotatif et la charge est linéaire)
		\end{itemize}
	
		\subsection*{Réducteurs rotatif-rotatif}
		
		
			\[i = \frac{\omega_M}{\omega_L} = \frac{Z_L}{Z_M} = \frac{\Delta \theta_M}{\Delta \theta_L} = \frac{\alpha_M}{\alpha_L} \]
		
			\begin{itemize}
				\item i: rapport de réduction
				\item $\omega_M$, $\omega_L$ : vitesses du moteur, respectivement de la charge
				\item $Z_M$, $Z_L$ nombres de dents des pignons côté moteur, respectivement côté charge
			\end{itemize}
		
		\subsubsection*{Rendement}
		
			\[\eta = \frac{P_{utile}}{P_{fournie}} \leq 1.00 \]
			\[P_M = \omega_M \cdot T_M,\; P_L= \omega_L \cdot T_L \]
			
			En régime moteur:
			\[\eta_{M \rightarrow L} = \frac{P_L}{P_M} \]
			\[i = \frac{T_L}{\eta_{M \rightarrow L} \cdot T_M} \]
			
			En régime générateur / frein:
			\[\eta_{L \rightarrow M} = \frac{P_M}{P_L} \]
			\[i = \frac{\eta_{L \rightarrow M} \cdot T_L}{T_M} \]
		
		\subsection*{Réducteurs rotatifs-linéaires}
		
			\[i = \frac{\omega_M}{v_L} = \frac{2 \cdot \pi}{Z_M \cdot p} \]
			\[P_L = v_L \cdot F_L \]
			
			En régime moteur:
			\[i = \frac{F_L}{\eta_{M \rightarrow L} \cdot T_M} \]
			
			En régime générateur / frein:
			\[i = \frac{\eta_{L \rightarrow M} \cdot F_L}{T_M} \]
			
			Pour un treuil:
			\[v_L = r \cdot \omega_M \]

		\subsection*{Choix du rapport de réduction - Régime permanent}
	
			Contrainte de vitesse:
			\[i<i_{\max }=\frac{\omega_{M-l i m}}{\omega_{L-\max }} \text{ resp. } i<i_{\max }=\frac{\omega_{M-l i m}}{v_{L-\max }}\]
			
			Contrainte de couple en régime «moteur»:
			\[i>i_{\min }=\frac{T_{L-\max }}{T_{M-n o m}} \cdot \frac{1}{\eta} \text{ resp. } i>i_{\min }=\frac{F_{L-\max }}{T_{M-n o m}} \cdot \frac{1}{\eta}\left[\mathrm{m}^{-1}\right]\]
			
			Contrainte de couple en régime générateur/frein:
			\[i>i_{\min }=\frac{T_{L-\max }}{T_{M-n o m}} \cdot \eta \text{ resp. } i>i_{\min }=\frac{F_{L-m a x}}{T_{M-n o m}} \eta\left[\mathrm{m}^{-1}\right]\]
		
		\subsection*{Choix du rapport de réduction - Régime impulsionnel}
			
			Couple nécessaire pour accélérer le moteur+charge:
			
			\[\left.T_{a c c}\right|_{M}=\alpha_{M} \cdot \sum J=\alpha_{M} \cdot\left(J_{M}+\left.J_{L-equiv}\right|_{M}\right)\]
			
			Pour un réducteur rotatif-rotatif:
			\[\left.J_{L-equiv}\right|_{M}=J_{L} \cdot\left(\frac{1}{i}\right)^{2}=J_{L} \cdot\left(\frac{Z_{M}}{Z_{L}}\right)^{2}\left[\mathrm{kgm}^{2}\right]\]
			\[i_{o p t}=\sqrt{\frac{J_{L}}{J_{M}}} \quad \text { (sans dimension) }\]
			
			Pour un réducteur rotatif-linéaire:
			\[\left.J_{L-equiv}\right|_{M}=m_{L} \cdot\left(\frac{1}{i}\right)^{2}=m_{L} \cdot\left(\frac{Z_{M} \cdot p}{2 \cdot \pi}\right)^{2}\left[\mathrm{kgm}^{2}\right]\]
			\[i_{o p t}=\sqrt{\frac{m_{L}}{J_{M}}} \quad\left[\mathrm{~m}^{-1}\right]\]

	\section*{Moteurs électriques}
	
		\subsection*{Moteurs à courant continu (DC)}
		
			\subsubsection*{Équation de conversion e.m.}
			
				Constante de couple $k_T$:
				\[T_{e m}=k_{T} \cdot I_{a}\]
				
				Constante de vitesse $k_E$:
				\[U_{i}=k_{E} \cdot \omega\]
				
				En négligeant les pertes:
				\[k_{T}=k_{E}\]
			
			\subsubsection*{Équation électrique}
			
				Cas général:
			
				\[u_{0}(t)=\underbrace{\left(R_{a}+R_{i}\right)}_{R_{\text {total }}} \cdot i_{a}(t)+\underbrace{\left(L_{a}+L_{i}\right)}_{L_{\text {total }}} \cdot \frac{d i_{a}(t)}{d t}+u_{i}(t)\]
				
				En négligeant l'effet de l'alimentation:
				
				\[u_{a}(t)=R_{a} \cdot i_{a}(t)+L_{a} \cdot \frac{d i_{a}(t)}{d t}+u_{i}(t)\]
				
				\begin{figure}[H]
					\includegraphics[width=8cm]{img/modele_moteur_dc.png}
					\centering
				\end{figure}
				
				En régime permanent:
				
				\[U_{a}=R_{a} \cdot I_{a}+U_{i}=R_{a} \cdot I_{a}+k_{\varepsilon} \cdot \omega\]
				\[\frac{U_{i}}{T_{e m}}=\frac{k_{E} \cdot \omega}{k_{T} \cdot I_{a}} \Rightarrow \frac{U_{i} \cdot I_{a}}{T_{e m} \cdot \omega}=\frac{k_{E}}{k_{T}}\]
				
			\subsubsection*{Équation cinématique}
				
				\[T_{e m}(t)-\underbrace{\left[T_{\text {frott }-M}(t)+T_{\text {frott }-L}(t)+T_{\text {utile }}(t)\right]}_{T_{\text {res }}(t)}=T_{a c c}(t)\]
				\[\underbrace{T_{e m}(t)-T_{r é s}(t)}_{T_{a c}(t)}=J_{t o t a l} \cdot \frac{d \omega(t)}{\frac{d t}{\alpha(t)}}\]
				\[J_{\text {total }}=J_{M}+J_{L \text { -équiv }} \mid\]
				
			\subsubsection*{Puissance et rendement}
				
				Puissance mécanique:
				
				\[P_{\text {arbre }}(t)=T_{\text {arbre }}(t) \cdot \omega(t)\]
				
				Puissance électrique:
				\[P_{\text {élec }}(t)=u_{a}(t) \cdot i_{a}(t)\]
				
				En mode moteur:
				\[\eta=\frac{P_{\text {utile }}(t)}{P_{\text {faurnie }}(t)}=\frac{P_{\text {arbre }}(t)}{P_{\text {élec }}(t)}\]
				
				En mode génératrice:
				\[\eta=\frac{P_{\text {utile }}(t)}{P_{\text {fournie }}(t)}=\frac{P_{\text {élec }}(t)}{P_{\text {arbre }}(t)}\]
				
				Pertes électriques:
				\[P_{\text {Joule }}(t)=R_{a} \cdot i_{a}^{2}(t)\]
				
				Pertes mécaniques:
				\[P_{\text {frott }}(t)=T_{\text {frott. }}(t) \cdot \omega(t)\]
				
				Puissance électromagnétique:
				\[P_{e m}(t)=P_{\text {élec }}(t)-P_{\text {Joule }}(t)=P_{\text {arbre }}(t)+P_{\text {frott }}(t)\]
			
			\subsubsection*{Comportement dynamique des moteurs DC}
				
				Constante de temps électrique:
				\[\tau_{é l}=\frac{L_{a}}{R_{a}}\]
				
				En prenant en compte de l'effet de l'alimentation:
				\[\tau_{é l}=\frac{L_{t o t a l}}{R_{t o t a l}}=\frac{L_{a}+L_{i}}{R_{a}+R_{i}}\]
				
				Constante de temps mécanique:
				\[\tau_{m é c}=\frac{R_{a} \cdot J_{t o t a l}}{k_{T} \cdot k_{E}}\]
				
				Avec $\tau_{élec} << \tau_{mec}$:
				
				\[\tau_{m é c} \cdot \frac{d \omega(t)}{d t}+\omega(t)-\frac{U_{a}}{k_{E}}=0\]
				
				\[\omega(\mathrm{t})=\Omega_{0}+\left(\Omega_{\infty}-\Omega_{0}\right) \cdot\left(1-\mathrm{e}^{-\mathrm{t} / \tau_{\mathrm{méc}}}\right)\]
				
				\[\mathrm{I}_{\text {initial }}=\frac{\mathrm{U}_{\mathrm{a}}}{\mathrm{R}_{\mathrm{a}}}\]
				
				\[i(t)=I_{0}+\left(I_{\infty}-I_{0}\right) \cdot\left(1-\mathrm{e}^{-t / \tau_{e l}}\right)\]
				
			\subsubsection*{Variantes d'excitation des moteurs DC}
				
				Moteur à excitation séparée:
				
				\[T_{e m}(t)=\left[k \cdot i_{e}(t)\right] \cdot i_{a}(t)\]
				\[u_{i}(t)=\left[k \cdot i_{e}(t)\right] \cdot \omega(t)\]
				
				Moteur à excitation série:
				
				\[T_{e m}(t)=k \cdot\left[i_{a}(t)\right]^{2}\]
				
		\subsection*{Moteurs synchrones}
		
			\subsubsection*{Principe de fonctionnement}
				
				\[N=\frac{60 \cdot f}{p} \quad[\mathrm{tr} / \mathrm{min}] \quad \text { ou } \quad \omega=\frac{2 \pi \cdot f}{p} \quad[\mathrm{rad} / \mathrm{s}]\]
			
			\subsubsection*{Puissance}
				
				Puissance mécanique:
				\[P_{\text {arbre }}(t)=T_{\text {arbre }}(t) \cdot \omega(t)\]
				
				Puissance électrique:
				\[P_{\text {élec }}(t)=\sqrt{3} \cdot U_{c} \cdot i(t) \cdot \cos [\varphi(t)]\]
		
		\subsection*{Moteurs asynchrones}
		
			\subsubsection*{Généralités}
				
				Glissement:
				\[s=\frac{N_{s}-N}{N_{s}}\]
				
			\subsubsection*{Puissance}
				Puissance à l'arbre:
				\[P_{\text {arbre }}(t)=T_{\text {arbre }}(t) \cdot \omega(t)\]
				
				Puissance électrique:
				\[P_{\text {élec }}(t)=\sqrt{3} \cdot U_{c} \cdot i(t) \cdot \cos [\varphi(t)]\]
				
				Puissance mécanique:
				\[P_{m é c}(t)=T_{r é s}(t) \cdot \omega(t)\]
			
			\subsubsection*{Régime de survitesse}
			
			Limite de fonctionnement à puissance constante:
			\[T_{\text {survitesse }} \cdot \omega_{\text {survitesse }}<T_{\text {nom }} \cdot \omega_{\text {nom }}\]
			
		\subsection*{Moteurs pas-à-pas}
			
			\subsubsection*{Principe de fonctionnement}
				
				\[\omega(t)=f_{\text {pulse }}(t) \cdot \frac{2 \cdot \pi}{p_{\text {step }}} \quad[\mathrm{rad} / \mathrm{s}] \quad \text { ou } \quad N(t)=f_{\text {pulse }}(t) \cdot \frac{60}{p_{\text {step }}} \quad[\mathrm{tr} / \mathrm{min}]\]
				
		\subsection*{Autres types de moteurs électriques}
		
			\begin{itemize}
				\item Électroaimants
				\item Moteurs à bobine mobile
				\item Moteurs linéaires et moteurs couples
				\item Moteurs linéaires «piston»
				\item Piézoactionneurs et piézomoteurs
			\end{itemize}
		
		\subsection*{Choix d'un moteur électrique}
			
			\subsubsection*{Fonctionnement en régime permanent}
			
			\subsubsection*{Fonctionnement en régime impulsionnel}
				
				Dimensionnement thermique - 1ère hypothèse:
				\[\Delta \vartheta=P_{\text {therm }} \cdot R_{\text {therm }}\]
				
				Dimensionnement thermique - 2ème hypothèse:
				\[P_{\text {pertes totales }}(t)=k_{\text {therm }} \cdot T_{e m}^{2}(t)\]
				
				Évolution de la température en fonction du couple:
				\[\Delta \vartheta_{\text {nom }}=P_{\text {therm nom }} \cdot R_{\text {therm }}\]
				\[\Delta E_{\text {therm }}=m \cdot c_{m} \cdot \Delta \vartheta\]
				\[\Delta \vartheta(t)=\Delta \vartheta_{0}+\left(\Delta \vartheta_{\infty}-\Delta \vartheta_{0}\right) \cdot\left(1-e^{-\frac{t}{\tau_{t h e r m}}}\right)\]
				\[\Delta \vartheta_{\infty}=P_{\text {therm }} \cdot R_{\text {therm }}\]
				\[\tau_{\text {therm }}=m \cdot c_{m} \cdot R_{\text {therm }}\]
				\[\frac{\Delta \vartheta(t)}{\Delta \vartheta_{\text {nom }}}=\frac{P_{\text {therm }}}{P_{\text {therm nom }}} \cdot\left(1-e^{-\frac{t}{\tau_{\text {therm }}}}\right)\]
				\[\frac{\Delta \vartheta(t)}{\Delta \vartheta_{n o m}}=\left(\frac{T_{e m}}{T_{n o m}}\right)^{2} \cdot\left(1-e^{-\frac{t}{\tau_{\text {therm }}}}\right) \leq 1\]
				\[t_{c y c l e}=\frac{3^{\prime} 600}{C}\]
				\[T_{r . m . s .}=\sqrt{\frac{\int_{0}^{t_{c y c l e}} T_{e m}^{2}(t) \cdot d t}{t_{c y c l e}}}<0,9 \cdot T_{n o m}\]
				\[T_{r . m . s .}=\sqrt{\frac{T_{e m-1}^{2} \cdot t_{1}+T_{e m-2}^{2} \cdot t_{2}+\cdots+T_{e m-n}^{2} \cdot t_{n}}{t_{c y c l e}}}<0,9 \cdot T_{n o m}\]
				\[\text { marge }=1-\frac{T_{r . m . s .}}{T_{n o m}}\]
				
	\section*{Alimentation des moteurs}
		
		%\subsection*{Précurseurs des convertisseurs électroniques}
		
		%\subsection*{Amplificateurs analogiques}
		
		\subsection*{Amplificateurs à découpage}
			
			\[U_{M-m o y e n}=U_{A} \cdot \underbrace{\frac{t_{e}}{t_{c y c l e}}}_{=\delta}\]
			\[P_{S} \cong\left[1,0+U_{A} \cdot\left(t_{o f f-o n}+t_{o n-o f f}\right) \cdot f_{d}\right] \cdot I_{M}\]
			
		\subsection*{Considérations d'énergie et de puissance}
			
			Alimentation triphasée:
			\[P_{A C}(t)=\sqrt{3} \cdot U_{c o m p} \cdot I_{A C}(t)\]
			
			Bus DC:
			\[P_{D C}(t)=U_{D C} \cdot I_{D C}(t)\]
			
			Bornes moteur:
			\[P_{M-\mathrm{e} l}(t)=U_{M}(t) \cdot I_{M}(t)\]
			
			Arbre moteur:
			\[P_{M-m é c}(t)=\omega_{M}(t) \cdot T_{M}(t)\]
			
			\[P_{A C}(t) \cong P_{D C}(t) \cong P_{M-\mathrm{e} l}(t) \cong P_{M-m é c}(t)\]
			
		\subsection*{Freinage des servo-moteurs}
			
			\[C_{\text {total }}>\frac{2 \cdot E_{c i n}}{U_{A-l i m}^{2}-U_{A-m a x-Q 1}^{2}}\]
			\[R_{\text {frein }}<\frac{U_{D C-\text { seuil }}^{2}}{P_{\text {frein-max }}}\]
			\[P_{\text {frein-moyen }}=\frac{\frac{1}{2} \cdot J_{\text {total }} \cdot \omega_{\max }^{2}}{t_{\text {cycle }}}\]
			
	\section*{Pilotage en position}
		
		%\subsection*{Réglage en position}
		
		\subsection*{Profils de mouvement}
			
			\subsubsection*{Profil à vitesse constante}
				
				\[p(t)=t\]
				
			\subsubsection*{Profil à accélération constante}
				
				\[
				p(t)=\left\{\begin{array}{ll}
				2 \cdot t^{2} & \text { si } t \leq 0,5 \\
				1-2 \cdot(1-t)^{2} & \text { si } t>0,5
				\end{array}\right.
				\]
				
			\subsubsection*{Profil «bang-bang»}
			
				\[
				p(t)=\left\{\begin{array}{cc}
				\frac{9}{4} \cdot t^{2} & \text { si } t \leq \frac{1}{3} \\
				\frac{1}{4}+\frac{3}{2} \cdot\left(t-\frac{1}{3}\right) & \text { si } t>\frac{1}{3 \text { et } t} \leq \frac{2}{3} \\
				1-\frac{9}{4} \cdot(1-t)^{2} & \text { si } t>\frac{2}{3}
				\end{array}\right.
				\]
				
			\subsubsection*{Loi semi-sinusoïdale}
				
				\[p(t)=\frac{1-\cos (\pi \cdot t)}{2}\]
				
			\subsubsection*{Loi sinusoïdale}
				
				\[p(t)=t-\frac{\sin (2 \pi \cdot t)}{2 \pi}\]
				
			\subsubsection*{Polynôme du 5ème degré}
			
				\[p(t)=10 \cdot t^{3}-15 \cdot t^{4}+6 \cdot t^{5}\]
	
	\end{flushleft}
	\end{multicols}
\end{document}