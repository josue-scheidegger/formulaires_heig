\documentclass[12pt, twocolumn]{article}
\usepackage[landscape, a4paper, margin=2cm]{geometry}
\usepackage[utf8]{inputenc}
\usepackage[T1]{fontenc}
\usepackage{textcomp}
\usepackage{graphicx}
\usepackage{float}
\usepackage{amsmath}

\title{MotComand - Formulaire}

\begin{document}
	\maketitle
	
		\section*{Mouvement dans les machines}
		
			\subsection*{Actionneurs et moteurs}
	
				\begin{itemize}
					\item \emph{Actionneurs et moteurs pneumatiques:} économiques, faible coûts d'entretien, conviennent aux milieux hostiles, vitesses élevées; temps de réaction < 20ms, bruit, positions limitées (tout-ou-rien).
					\item \emph{Actionneurs et moteurs hydrauliques:} performants, haute densité d'énergie, réglage en vitesse ou en position; coûteux, entretien plus compliqué (huile), temps de réponse d'environ 2ms.
					\item \emph{Moteurs électriques:} économiques, beaucoup de fournisseurs, faciles à mettre en oeuvre, temps de réponse de 0.1ms; nécessitent en général des réducteurs.
				\end{itemize}
				
				Comparaison:
				
				\begin{figure}[H]
					\includegraphics[width=8cm]{img/comparaison_actionneurs.png}
					\centering
				\end{figure}
			
			\subsection*{Constitution des entraînements}
				
				\begin{figure}[H]
					\includegraphics[width=8cm]{img/constitution_entrainements.png}
					\centering
				\end{figure}
				
			\subsection*{Charges}
				
				Lois de Newton:
				\[a = \frac{\sum F}{m}\]
				
				\begin{itemize}
					\item a : accélération $[m/s^2]$
					\item F : forces $[N]$
					\item m : masse $[kg]$
				\end{itemize}
		
				\[\alpha = \frac{\sum T}{J} \]
		
				\begin{itemize} 
					\item $\alpha$ : accélération angulaire $[rad/s^2]$
					\item T : couples $[N \cdot m]$
					\item J : inertie $[kg \cdot m^2]$
				\end{itemize}
		
				Inertie d'un cylindre:
				
				\[ J = \frac{m \cdot R^2}{2} = \frac{\rho \cdot L \cdot \pi \cdot R^4}{2} \]
				
				Quadrants de fonctionnement:
				
				\begin{table}[h!]
				\begin{tabular}{|c|c|c|l|}
				\hline
					& $\omega$ & $T_{em}$ & Mode   \\ \hline
				1 & +        & +        & Moteur \\ \hline
				2 & -        & +        & Frein  \\ \hline
				3 & -        & -        & Moteur \\ \hline
				4 & +        & -        & Frein  \\ \hline
				\end{tabular}
				\centering
				\end{table}
				
				Types de charge:
				
				\begin{itemize}
					\item Charge à couple constant
					\item Charge à couple croissant avec la vitesse
					\item Charge à puissance constante: $P(t)=T(t) \cdot \omega(t)=[F \cdot r(t)] \cdot\left[\frac{V}{r(t)}\right]=F \cdot V=$ constante (avec F et V, respectivement force et vitesse tangentielles)
				\end{itemize}
				
				Régimes de fonctionnement:
				
				\[T_{moteur}(t) - T_{resistant}(t) = T_{accel.}(t) \]
				\[T_{resistant}(t) = T_{frott.}(t) + T_{utile}(t) \]
				
				\begin{itemize}
					\item \emph{Régime permanent:} la charge tourne à vitesse (quasi) constante
					\item \emph{Régime impulsionnel ou intermittent:} la charge est constamment accélérée et freinée
				\end{itemize}
	
			\subsection*{Modes de fonctionnement}
	
				\begin{itemize}
					\item \emph{Mode tout-ou-rien:} le plus simple et le plus bon marché; pas d'adaptation à la charge entraînée.
					\item \emph{Mode contrôlé en boucle ouverte:} contrôle approximatif de la vitesse et de l'effort fournis.
					\item \emph{Mode contrôlé en «boucle fermée»:} grande précision; plus complexe et coûteux.
					\item \emph{Mode servomoteur - réglé en position:} permet de contrôler tous les mouvements d'une machine; complexes et coûteux.
					\item \emph{Mode pas-à-pas:} simple et bon marché; limité en puissance (~200W) et vitesse (~1000tr/min).
				\end{itemize}
	
		
	
			\subsection*{Modèle thermique des moteurs}
	
				\[\Delta T = R_{th} \cdot P_{moy} \]
				\[P_c = C_{th} \cdot \frac{dT}{dt} \]
				\[T(t) = (T_{max} - T_0) \cdot (1 - e^{-t/\tau_{th}}) + T_0 \]

				Comparaison thermique-électricité:
	
				\begin{itemize}
					\item Courant électrique [A] = puissance thermique [W]
					\item Tension électrique [V] = température [°C]
					\item Capacité électrique = capacité thermique
				\end{itemize}
	
	\section*{Réducteurs}
		
		Types de réducteurs:
		
		\begin{itemize}
			\item réducteurs rotatifs-rotatifs (le moteur et la charge sont rotatifs)
			\item réducteurs rotatifs-linéaire (le moteur est rotatif et la charge est linéaire)
		\end{itemize}
	
		\subsection*{Réducteurs rotatif-rotatif}
		
		
			\[i = \frac{\omega_M}{\omega_L} = \frac{Z_L}{Z_M} = \frac{\Delta \theta_M}{\Delta \theta_L} \]
		
			\begin{itemize}
				\item i: rapport de réduction
				\item $\omega_M$, $\omega_L$ : vitesses du moteur, respectivement de la charge
				\item $Z_M$, $Z_L$ nombres de dents des pignons côté moteur, respectivement côté charge
			\end{itemize}
		
		\subsubsection*{Rendement}
		
			\[\eta = \frac{P_{utile}}{P_{fournie}} \leq 1.00 \]
			\[P_M = \omega_M \cdot T_M,\; P_L= \omega_L \cdot T_L \]
			
			En régime moteur:
			\[\eta_{M \rightarrow L} = \frac{P_L}{P_M} \]
			\[i = \frac{T_L}{\eta_{M \rightarrow L} \cdot T_M} \]
			
			En régime générateur / frein:
			\[\eta_{L \rightarrow M} = \frac{P_M}{P_L} \]
			\[i = \frac{\eta_{L \rightarrow M} \cdot T_L}{T_M} \]
		
		\subsection*{Réducteurs rotatifs-linéaires}
		
			\[i = \frac{\omega_M}{v_L} = \frac{2 \cdot \pi}{Z_M \cdot p} \]
			\[v_L = r \cdot \omega_M \]
			\[P_L = v_L \cdot F_L \]
			
			En régime moteur:
			\[i = \frac{F_L}{\eta_{M \rightarrow L} \cdot T_M} \]
			
			En régime générateur / frein:
			\[i = \frac{\eta_{L \rightarrow M} \cdot F_L}{T_M} \]
			
			
			Pour un treuil:
			\[v_L = r \cdot \omega_M \]

		\subsection*{Choix du rapport de réduction - Régime permanent}
	
			Contrainte de vitesse:
			\[i<i_{\max }=\frac{\omega_{M-l i m}}{\omega_{L-\max }} \text{ respectivement } i<i_{\max }=\frac{\omega_{M-l i m}}{v_{L-\max }}\]
			
			Contrainte de couple en régime «moteur»:
			\[i>i_{\min }=\frac{T_{L-\max }}{T_{M-n o m}} \cdot \frac{1}{\eta} \text{ respectivement } i>i_{\min }=\frac{F_{L-\max }}{T_{M-n o m}} \cdot \frac{1}{\eta}\left[\mathrm{m}^{-1}\right]\]
			
			Contrainte de couple en régime générateur/frein:
			\[i>i_{\min }=\frac{T_{L-\max }}{T_{M-n o m}} \cdot \eta \text{ respectivement } i>i_{\min }=\frac{F_{L-m a x}}{T_{M-n o m}} \eta\left[\mathrm{m}^{-1}\right]\]
		
		\subsection*{Choix du rapport de réduction - Régime impulsionnel}
		
			\[\left.J_{L-equiv}\right|_{M}=J_{L} \cdot\left(\frac{1}{i}\right)^{2}=J_{L} \cdot\left(\frac{Z_{M}}{Z_{L}}\right)^{2}\left[\mathrm{kgm}^{2}\right]\]
			\[\left.J_{L-equiv}\right|_{M}=m_{L} \cdot\left(\frac{1}{i}\right)^{2}=m_{L} \cdot\left(\frac{Z_{M} \cdot p}{2 \cdot \pi}\right)^{2}\left[\mathrm{kgm}^{2}\right]\]
			\[\left.T_{a c c}\right|_{M}=\alpha_{M} \cdot \sum J=\alpha_{M} \cdot\left(J_{M}+\left.J_{L-equiv}\right|_{M}\right)\]
			
			Pour un réducteur rotatif-rotatif:
			\[i_{o p t}=\sqrt{\frac{J_{L}}{J_{M}}} \quad \text { (sans dimension) }\]
			
			Pour un réducteur rotatif-linéaire:
			\[i_{o p t}=\sqrt{\frac{m_{L}}{J_{M}}} \quad\left[\mathrm{~m}^{-1}\right]\]

	\section*{Moteurs DC}
		
		\[T_{e m}=k_{T} \cdot I_{a}\]
		\[U_{i}=k_{E} \cdot \omega\]
		\[U_{a}=R_{a} \cdot I_{a}+U_{i}=R_{a} \cdot I_{a}+k_{\varepsilon} \cdot \omega\]
		\[\frac{U_{i}}{T_{e m}}=\frac{k_{E} \cdot \omega}{k_{T} \cdot I_{a}} \Rightarrow \frac{U_{i} \cdot I_{a}}{T_{e m} \cdot \omega}=\frac{k_{E}}{k_{T}}\]
		\[k_{T}=k_{E}\]
		\[T_{e m}(t)-\underbrace{\left[T_{\text {frott }-M}(t)+T_{\text {frott }-L}(t)+T_{\text {utile }}(t)\right]}_{T_{\text {res }}(t)}=T_{a c c}(t)\]
		\[\underbrace{T_{e m}(t)-T_{r é s}(t)}_{T_{a c}(t)}=J_{t o t a l} \cdot \frac{d \omega(t)}{\frac{d t}{\alpha(t)}}\]
		\[P_{\text {arbre }}(t)=T_{\text {arbre }}(t) \cdot \omega(t)\]
		
		En mode moteur:
		\[\eta=\frac{P_{\text {utile }}(t)}{P_{\text {faurnie }}(t)}=\frac{P_{\text {arbre }}(t)}{P_{\text {élec }}(t)}\]
		
		En mode génératrice:
		\[\eta=\frac{P_{\text {utile }}(t)}{P_{\text {fournie }}(t)}=\frac{P_{\text {élec }}(t)}{P_{\text {arbre }}(t)}\]
		
		Pertes électriques:
		\[P_{\text {Joule }}(t)=R_{a} \cdot i_{a}^{2}(t)\]
		
		Pertes par frottement:
		\[P_{\text {frott }}(t)=T_{\text {frott. }}(t) \cdot \omega(t)\]
		
		Puissance électromagnétique:
		\[P_{e m}(t)=P_{\text {élec }}(t)-P_{\text {Joule }}(t)=P_{\text {arbre }}(t)+P_{\text {frott }}(t)\]
\end{document}