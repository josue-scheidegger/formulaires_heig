\documentclass[10pt, twocolumn]{article}
\usepackage[landscape, a4paper, margin=2cm]{geometry}
\usepackage{amsmath}

\title{Formulaire TechMes}

\begin{document}
	
	\begin{center}
		{\Large \bf Formulaire TechMes} \\
	\end{center}
	
	\section*{Généralités}
		
		\subsection*{Méthodes de mesure}
		
			\begin{itemize}
				\item Par déviation
				\item Par comparaison
			\end{itemize}
			
		\subsection*{Unités fondamentales}
			
			\begin{enumerate}
				\item Masse en kilogramme [kg]
				\item Distance en mètre [m]
				\item Temps en seconde [s]
				\item Courant en ampère [A]
				\item Température en kelvin [K]
				\item Intensité lumineuse en candela [cd]
				\item Quantité de matière en mole [mol]
			\end{enumerate}
			
		\subsection*{Mesure d'une grandeur physique}
			
			\[G = g \pm \Delta g \; U \text{ à } x\% \text{ (ou }n-\sigma)\]
			
			\begin{itemize}
				\item $G$: nom de variable (m, I, t)
				\item $g$: valeur numérique
				\item $\Delta g$: incertitude
				\item $U$: unité
				\item $x\%$: probabilité que la vraie valeur de $G$ soit comprise dans l'intervalle $[g-\Delta g;g+\Delta g]$
				\item $n-\sigma$: intervalle de confiance
			\end{itemize}
			
			Exemple: $m=12.3\cdot 10^{-6} \pm 0.5\cdot 10^{-6} \; g \text{ a } 95\%$
			
	\section*{Chaîne de mesure}
	
		\subsection*{Généralités}
	
			\begin{enumerate}
				\item \emph{Mesurande:} grandeur non-électrique que l'on souhaite mesurer
				\item \emph{Transducteur (capteur):} signaux analogiques
				\item \emph{Conditionneur:} signaux continus dans le temps
				\item \emph{Pré-traitement:} amplification, filtrage
				\item \emph{Convertisseur A/N:} échantillonnage (discrétisation)
					\begin{enumerate}
						\item $x(t) \rightarrow$ Échantillonneur
						\item $x[n] \rightarrow$ Quantificateur
						\item Codeur $\rightarrow$ signal numérique
					\end{enumerate}
				\item \emph{Post-traitement et stockage des données}
				\item \emph{Résultat}
			\end{enumerate}
		
		\subsection*{Transducteurs: capteurs et actionneurs}
			
			\begin{description}
				\item [Transducteur:] conversion d'une grandeur physique en une autre.
				\item [Capteur:] conversion d'une grandeur physique en un signal électrique.
				\item [Actionneur:] génère une grandeur physique depuis un signal électrique.
				\item [Grandeurs d'influence:] grandeurs d'entrée non désirées du système.
			\end{description}
			
			Grandeurs d'influence les plus courantes: 
			
			\begin{itemize}
				\item Température
				\item Tension d'alimentation
				\item Temps
				\item Humidité relative
			\end{itemize}
		
		\subsection*{Problème de mesure}
		
			\[Y = F(X) \quad X_m=F^{-1}(Y)\]
			
			\begin{itemize}
				\item $X$ : vraie valeur du mesurande (que l'on ne connaîtra jamais exactement)
				\item $X_m$ : valeur mesurée; est une estimation de $X$
				\item \emph{Validité des mesures:} degré de confiance que l'on peut accorder au résultat chiffré de la mesure.
			\end{itemize}
			
			Causes d'erreurs:
		
			\begin{itemize}
				\item modèle mathématique (non-conformité, non-linéarité)
				\item effet des grandeurs d'influence (modification du comportement de la chaîne)
				\item bruit interne (limite de détection)
				\item perturbations provoquées par l'environnement externe (compatibilité électromagnétique)
				\item effet de charge (échange d'énergie entre l'objet mesuré et la chaîne de mesure)
			\end{itemize}
			
		\subsection*{Modèle mathématique}
			
			Modèle linéaire:
			\[Y = G \cdot X + \text{Offset}\]
			
			Modèle polynomial:
			\[Y=a_0+a_1\cdot X+a_2\cdot X^2+...+a_n\cdot X^n\]
			\[X_{m}=c_{0}+c_{1} \cdot Y+c_{2} \cdot Y^{2}+\ldots+c_{n} \cdot Y^{n}\]
		
	\section*{Etalonnage, ajustage}
		\subsection*{Incertitudes}
		
		\subsection*{Ajustage et étalonnage}
			
			\begin{description}
				\item [Ajustage:] modifier le comportement => meilleure réponse possible (la plus proche des valeurs nominales).
				\item [Étalonnage:] déterminer les erreurs actuelles.
			\end{description}
			
			\subsubsection{Incertitudes résiduelles d'ajustage}
			
				Calcul du gain:
				\[G_{n}=\frac{Y_{\max }-Y_{o}}{X_{\max }-X_{o}}\]
				
				Incertitude de décalage:
				\[D_{c}=\Delta X_{o}+\mid \frac{\Delta Y_{0}}{G_{n}}\]
				
				Incertitude de gain:
				\[\alpha_{c}=\frac{D_{c}+\Delta X_{\max }+\left|\frac{\Delta Y \max }{G_{n}}\right|}{X_{\max }-X_{o}}\]
			
			\subsection{Comparer les modèles actuels et nominals}
			
				Erreur de gain:
				\[\alpha=\frac{G_{r}-G_{n}}{G_{n}}\]
				
				Erreur de décalage:
				\[D=\frac{O f_{r}-O f_{n}}{G_{n}}\]
			
			\subsection{Non-linéarité}
			
			Écart entre la courbe et le modèle réel:
			\[\Delta \mathrm{i}=\mathrm{Yi}-(\mathrm{GrXi}+\mathrm{Ofr})\]
			\[\delta \mathrm{i}=\mathrm{Xi}-(\mathrm{Yi}-\mathrm{Ofr}) / \mathrm{Gr}\]
			
			Spécification d'incertitude:
			\[\mathrm{NL}=\max \{|\Delta \mathrm{i}| / \mathrm{Gr}\}=\max \{|\delta \mathrm{i}|\}\]
			
		\subsection*{Compensation des erreurs systématiques}
		\subsection*{Mesures répétées}
			
			Sensiblité:
			\[K(M) = \frac{\partial C(M)}{\partial (M)}\]
			
			Fidélité d'un instrument:
			\[F = \sqrt{(valeur\_etalon - valeur\_mesuree)^{2}}\]
			
			Justesse d'un instrument:
			\[J [\%] = (1 - \frac{valeur\_etalon - valeur\_mesuree}{valeur\_etalon})\]
				% Ajouter la valeur absolue
			
			Théorème de Shanon-Nyquist:
			\[f_{e} \geq 2 \cdot f_{max}\]
			
			Quantum (résolution):
			\[q = \frac{X_{PE}}{2^N}\]
			\[X_{PE} = X_{max}-X_{min}\]
			
			Erreur de quantification:
			\[-q/2 \leq e_q < q/2\]
			
			Fréquence de scrutation et de balayage:
			\[f_{scan} \leq \frac{f_{scrut}}{N_{canaux}}\]
			
	\section*{Analyse de mesures}
	\section*{Capteurs}
	\section*{Lexique}
	
		\begin{description}
		
			\item [Résolution de mesure:] plus petite variation du mesurande que l'on peut détecter (1 quantum = 1 digit = 1 LSB)
		\end{description}
		
		
\end{document}