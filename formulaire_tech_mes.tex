\documentclass[10pt]{article} % twocolumn
\usepackage[landscape, a4paper, margin=2cm]{geometry}
\usepackage{amsmath}
\usepackage{multicol}
\usepackage[utf8]{inputenc}

\title{Formulaire TechMes}

\begin{document}
\begin{multicols}{3}
\begin{flushleft}
	
	\begin{center}
		{\Large \bf Formulaire TechMes} \\
	\end{center}
	
	\section*{Introduction}
		
		\subsection*{Méthodes de mesure}
		
			\begin{itemize}
				\item Par déviation
				\item Par comparaison
			\end{itemize}
			
		\subsection*{Unités fondamentales}
			
			\begin{enumerate}
				\item Masse en kilogramme [kg]
				\item Distance en mètre [m]
				\item Temps en seconde [s]
				\item Courant en ampère [A]
				\item Température en kelvin [K]
				\item Intensité lumineuse en candela [cd]
				\item Quantité de matière en mole [mol]
			\end{enumerate}
			
		\subsection*{Mesure d'une grandeur physique}
			
			\[G = g \pm \Delta g \; U \text{ à } x\% \text{ (ou }n-\sigma)\]
			
			\begin{itemize}
				\item $G$: nom de variable (m, I, t)
				\item $g$: valeur numérique
				\item $\Delta g$: incertitude
				\item $U$: unité
				\item $x\%$: probabilité que la vraie valeur de $G$ soit comprise dans l'intervalle $[g-\Delta g;g+\Delta g]$
				\item $n-\sigma$: intervalle de confiance
			\end{itemize}
			
			Exemple: $m=12.3\cdot 10^{-6} \pm 0.5\cdot 10^{-6} \; g \text{ a } 95\%$
			
	\section*{Chaîne de mesure}
	
		\subsection*{Introduction}
			
			\begin{enumerate}
				\item \emph{Chaîne de mesure:} succession d'appareils assurant la transmission et la transformation de l'information entre le capteur et le résultat de mesure.
				\item \emph{Mesurande:} grandeur d'entrée que l'on souhaite mesurer.
			\end{enumerate}
		
		\subsection*{Transducteurs: capteurs et actionneurs}
			
			\begin{description}
				\item [Transducteur (transducer):] conversion d'une grandeur physique en une autre.
				\item [Capteur (sensor):] conversion d'une grandeur physique en grandeur d'entrée du système.
				\item [Actionneur (actuator):] conversion d'une grandeur de sortie du système en grandeur physique.
				\item [Grandeurs d'influence $Z_i$:] grandeurs d'entrée non désirées du système.
			\end{description}
			
			Grandeurs d'influence les plus courantes: 
			
			\begin{itemize}
				\item Température
				\item Tension d'alimentation
				\item Temps
				\item Humidité relative
			\end{itemize}
		
		\subsection*{Résolution du problème de mesure}
		
			\[Y = F(X)\]
			\[X_m=F^{-1}(Y)\]
			
			\begin{itemize}
				\item $X$ : vraie valeur du mesurande (que l'on ne connaîtra jamais exactement)
				\item $X_m$ : valeur mesurée, qui est une estimation de $X$
				\item \emph{Validité des mesures:} degré de confiance que l'on peut accorder au résultat chiffré de la mesure.
			\end{itemize}
			
			Causes d'erreurs:
		
			\begin{itemize}
				\item modèle mathématique (non-conformité, non-linéarité)
				\item effet des grandeurs d'influence (modification du comportement de la chaîne)
				\item bruit interne (limite de détection)
				\item perturbations provoquées par l'environnement externe (compatibilité électromagnétique)
				\item effet de charge (échange d'énergie entre l'objet mesuré et la chaîne de mesure)
			\end{itemize}
			
		\subsection*{Modèle mathématique}
			
			\paragraph{Étendue de mesure:} domaine des valeurs du mesurande dans lequel le modèle mathématique est valable.
			
		\subsection*{Forme linéaire de la chaîne de mesure}
		
			\[Y = G \cdot X + \text{Offset}\]
			
		\subsection*{Forme polynomiale de la chaîne de mesure}
		
			\[Y=a_0+a_1\cdot X+a_2\cdot X^2+...+a_n\cdot X^n\]
			\[X_{m}=c_{0}+c_{1} \cdot Y+c_{2} \cdot Y^{2}+\ldots+c_{n} \cdot Y^{n}\]
			
%		\subsection*{Loi physique du capteur}
%		\subsection*{Erreurs de non-conformité ou de non-linéarité}
%		\subsection*{Bruit interne}
%		\subsection*{Grandeurs d'influence}
		\subsection*{Effet de charge}
			
			\begin{description}
				\item[Effet de charge (loading effect):] modification induites par la présence du transducteur.
			\end{description}
%		\subsection*{Perturbations}
%		\subsection*{Systèmes d'acquisition de données}
%		\subsection*{Architectures}
%		\subsection*{Modes du multiplexeur}
%		\subsection*{Sources flottantes et courants de polarisation}
%		\subsection*{Conditionnement des signaux}

		\subsection*{Différents types de signaux}
			
			\begin{itemize}
				\item Continu
				\item Discret
				\item Numérique
				\item Déterministes
				\item Aléatoires
			\end{itemize}
		
	\section*{Etalonnage, ajustage}
	
		%\subsection*{Approximations mathématiques}
	
		\subsection*{Incertitudes}
		
			Modèle linéaire:
			\[\mathbf{Y}=\mathbf{G} \mathbf{X}+\mathbf{O f}\]
			
			\begin{description}
				\item [Erreur absolue (e): ] écart entre la valeur mesurée et la vraie valeur.
				\item [Erreur relative ($\epsilon$):] quotient entre erreur absolue et vraie valeur.
				\item [Erreur de gain ($\alpha$):] erreur relative du gain de la chaîne. Erreur systématique proportionnelle au mesurande.
				\item [Erreur de décalage (D):] erreur absolue à l'origine. Erreur systématique constante, indépendante du mesurande.
				\item [Erreur de non-linéarité (NL):] écart entre la droite du modèle actuel et la courbe réelle de réponse.
				\item [Incertitude de mesure (I):] valeur limite que peut prendre l'erreur, avec un certain degré de confiance (en général 99\%).
			\end{description}
			
			\[e=X_m-X=X \cdot \alpha+D+N L\]
			\[\epsilon = \frac{e}{X} = \frac{e}{X_m}\]
			\[I=\alpha \cdot \text { lect }+B \cdot \text { digit }=\alpha \cdot \text { lect }+\beta \cdot \text { gamme }\]
			
		\subsection*{Ajustage et étalonnage}
			
			\begin{description}
				\item [Ajustage:] modifier le comportement pour une meilleure réponse possible.
				\item [Étalonnage:] déterminer les erreurs actuelles.
			\end{description}
			
			\subsubsection*{Ajustage}
					
					\begin{itemize}
						\item Modèle nominal: rectangle d'incertitude centré autour des valeurs nominales et contient les valeurs extrêmes mesurées.
						\item Modèle réel: idem que modèle nominal, mais n'est pas centré autour des valeurs moyennes; le rectangle est alors plus petit.
					\end{itemize}
					
					Gain nominal:
					\[G_{n}=\frac{Y_{\max }-Y_{o}}{X_{\max }-X_{o}}\]
					
					Incertitude de décalage:
					\[D_{c}=\Delta X_{o}+\mid \frac{\Delta Y_{0}}{G_{n}}\]
					
					Incertitude de gain:
					\[\alpha_{c}=\frac{D_{c}+\Delta X_{\max }+\left|\frac{\Delta Y \max }{G_{n}}\right|}{X_{\max }-X_{o}}\]
				
			\subsubsection*{Étalonnage}
				
				Erreur pour chaque couple mesuré:
				\[e(i)=X_{m}(i)-X(i)=\frac{Y(i)-O f_{\text {nom }}}{G_{\text {nom }}}-X(i)\]
				
				Méthodes possibles pour choix de la droite:
				\begin{description}
					\item [Droite par les extrêmes:] la droite passe par deux points de la réponse réelle.
					\item [Meilleure droite:] droite de régression linéaire.
				\end{description}
				
				Écart entre la valeur mesurée et la droite:
				\[\Delta=Y(i)-\left[G_{r e e l} \cdot X(i)+O f_{r e e l}\right]\]
				\[N L+\text { Bruit }=\frac{\max |\Delta|}{G_{\text {reel }}}\]
				
				Erreur de gain:
				\[\alpha=\frac{G_{r}-G_{n}}{G_{n}}\]
				
				Erreur de décalage:
				\[D=\frac{O f_{r}-O f_{n}}{G_{n}}\]
				
			\subsubsection*{Auto-calibrage}
			
		%\subsection*{Compensation des erreurs systématiques}
		
			%\subsubsection*{Compensation linéaire}
			%\subsubsection*{Moteur de correction mathématique}
		
		\subsection*{Mesures répétées}
			
			Moyenne:
			\[\mu=\lim _{N \rightarrow \infty} \frac{1}{N} \sum_{i=1}^{N} x_{i}\]
			
			Écart-type:
			\[\sigma=\lim _{N \rightarrow \infty} \sqrt{\frac{1}{N} \sum_{i=1}^{N}\left(x_{i}-\mu\right)^{2}}\]
			
			\begin{itemize}
				\item $1 \cdot \sigma \rightarrow 68 \%$
				\item $2 \cdot \sigma \rightarrow 95 \%$
				\item $3 \cdot \sigma \rightarrow 99.7 \%$
			\end{itemize}
			
			\begin{description}
				\item [Exactitude ou justesse:] caractérise l'erreur systématique, lié à la moyenne.
				\item [Répétabilité ou fidélité:] caractérise l'erreur aléatoire, lié à l'écart-type.
			\end{description}
			
%	\section*{Analyse de mesures}
%	
%		\subsection*{La mesure et sa représentation}
%			\subsubsection*{Définitions}
%			\subsubsection*{Moyenne, écart quadratique moyen, écart-type}
%			\subsubsection*{Représentation d'un échantillon de mesures à l'aide d'un histogramme}
%			\subsubsection*{Histogramme cumulé}
%			\subsubsection*{Médiane et mode}
%		\subsection*{La mesure vue comme stochastique}
%			\subsubsection*{Définitions}
%			\subsubsection*{Universalité des lois de probabilité}
%			\subsubsection*{Distribution de probabilité des variables aléatoires discrètes}
%			\subsubsection*{Distribution de probabilité des variables aléatoires continues}
%			\subsubsection*{Distribution de probabilité continues usuelles}
%			\subsubsection*{Incertitude de la moyenne empirique, intervalle et niveau de confiance}
%			
%		\subsection*{Mesures multidimensionnelles}
%			\subsubsection*{Covariance et corrélation}
%			\subsubsection*{Incertitude globale et corrélation des variables internes}
%			\subsubsection*{Cas particuliers fréquents}
%			\subsubsection*{Budget d'erreur}
%		\subsection*{Ajustement du modèle}
		
%	\section*{Capteurs}
%	
%		\subsection*{Classification des capteurs}
%		
%			\begin{itemize}
%				\item Principe de fonctionnement actif ou passif
%				\item Type de mesure absolue ou relative
%				\item Types de matériaux utilisés
%				\item Principes de mesure
%				\item Stimuli utilisé
%				\item Champs d'application
%			\end{itemize}
%			
%		\subsection*{Choix d'un capteur}
%		
%			\begin{table}[h!]
%			\begin{tabular}{|l|l|}
%			\hline
%			MESURANDE Conditions imposées               & CAPTEUR Caractéristiques métrologiques            \\ \hline
%			Plage de variation                          & Étendue de mesure                                 \\ \hline
%			Variation minimale à mesurer                & Résolution                                        \\ \hline
%			Spectre de fréquence ou vitesse de rotation & Bande passante                                    \\ \hline
%			Précision de mesure                         & Erreur de linéarité, erreur d’hystérésis          \\ \hline
%			Plage de température de fonctionne- ment    & Dérive thermique du zéro, tenue en température    \\ \hline
%			Localisation                                & Encombrement                                      \\ \hline
%			Composition de l’atmosphère                 & Inertie chimique, protection                      \\ \hline
%			Parasites                                   & Blindage, isolement ou non par rapport à la masse \\ \hline
%			\end{tabular}
%			\end{table}
%			
%			Principaux termes:
%			
%			\begin{table}[]
%			\begin{tabular}{|l|l|l|l|}
%			\hline
%			Français & Anglais & Définition & exemple \\ \hline
%			Sensibilité & Sensitivity & output variation / input variation & mV/g, mA/oC \\ \hline
%			Stabilité & Stability & Coefficient de variation selon une grandeur d’influence & stabilité en température \\ \hline
%			Précision & Accuracy & Somme de toutes les perturbations qui influencent la sortie du capteur &  \\ \hline
%			Étendue mesure & Span, range & Valeur max mesurable - valeur min mesurable &  \\ \hline
%			Résolution & Resolution & Plus petite variation du mesurande mesurable par le capteur &  \\ \hline
%			Sélectivité & Selectivity & S’applique à des capteurs biochimiques, sensibles à une molécule plus particulièrement par rapport à une autre molécule &  \\ \hline
%			Temps de réponse & Response time & Temps de propagation entre l’entrée et la sortie du capteur &  \\ \hline
%			Conditions environnementales & Environmental conditions & Décrit les plages de variation admissible pour les paramètres extérieurs & humidité,température,pression \\ \hline
%			Facteur de surcharge & Overload characteristics & Capacité à supporter un dépassement de la plage de mesure du mesurande &  \\ \hline
%			Linearité & Linearity &  &  \\ \hline
%			Hystérèse & Hysteresis &  &  \\ \hline
%			Zone morte & Dead band & Plage de valeur du mesurande pour la- quelle la sensibilité du capteur est nulle ou mauvaise &  \\ \hline
%			Durée de vie & Operating life & Durée pendant laquelle les caractéristiques du capteur sont observées & mtbf = mean time before failure \\ \hline
%			Taille & Size &  &  \\ \hline
%			Poids & Weight &  &  \\ \hline
%			Prix & Price &  &  \\ \hline
%			\end{tabular}
%			\end{table}
%			
%			Définitions:
%			
%			\begin{description}
%				\item [Capteur actif:] n'a pas besoin d'énergie additionnelle.
%				\item [Capteur passif:] nécessite un signal d'excitation.
%				\item [Capteur absolu:] produit une sortie en relation directe avec une échelle physique absolue indépendante des conditions de mesure.
%				\item [Capteur relatif:] sortie dépend du contexte.
%			\end{description}
%			
%		\subsection*{Matériaux utilisés}
%		
%			\begin{table}[]
%			\begin{tabular}{|l|l|}
%			\hline
%			Organique              & Matière fabriquée par les êtres vivants                              \\ \hline
%			Inorganique            & Matière qui ne possède pas les caractéristiques nécessaires à la vie \\ \hline
%			Conducteur             & Corps capable de transmettre de l’électricité                        \\ \hline
%			Isolant                & Matériau qui isole de l’électricité                                  \\ \hline
%			Semi-conducteur        &                                                                      \\ \hline
%			Liquide, gaz ou plasma & Etats de la matière                                                  \\ \hline
%			Substancebiologique    & Matériau extrait du monde vivant                                     \\ \hline
%			\end{tabular}
%			\end{table}
%			
%		\subsection*{Principes de conversion}
%			
%			\begin{itemize}
%				\item Physiques
%				\item Chimiques
%				\item Biologiques
%			\end{itemize}
%			
%		\subsection*{Mesurande}
%			
%			\begin{itemize}
%				\item Onde
%				\item Biologique
%				\item Chimique
%				\item Électrique
%				\item Magnétique
%				\item Onde optique
%				\item Mécanique
%				\item Radiation
%				\item Thermique
%			\end{itemize}
%		
%		\subsection*{Capteur réel}
%			
%			\begin{itemize}
%				\item Erreurs de décalage et de gain
%				\item Erreurs de linéarité et d'hystérèse
%				\item Erreurs de mobilité et de bruit
%				\item Erreurs systématiques et aléatoires
%			\end{itemize}
%			
%		\subsection*{Exemples}
%			
%			\begin{itemize}
%				\item Capteur de courant inductif
%				\item Capteur de courant par effet Hall
%				\item Température
%				\item Humidité (SHT3x-ARP de Sensirion)
%				\item Pression (Keller series 26 W)
%				\item Jauges de déformation
%				\item Force
%				\item Couple (Omega TQ513)
%				\item Position linéaire (LT1300)
%				\item Position angulaire (Baumer GM400)
%				\item Vitesse (Philips KMI15/1)
%				\item Capteur de vibration (Colibrys VS1000)
%				\item Capteur de proximité (Baumer CFDK 30N3600)
%				\item Capteur chimique (Membrapor CO/C-200)
%				\item Capteur optique (Hamamatsu S-4251)
%			\end{itemize}
		
		
%	\section*{Lexique}
%	
%		\begin{description}
%		
%			\item [Résolution de mesure:] plus petite variation du mesurande que l'on peut détecter (1 quantum = 1 digit = 1 LSB)
%		\end{description}
		
\end{flushleft}
\end{multicols}
\end{document}